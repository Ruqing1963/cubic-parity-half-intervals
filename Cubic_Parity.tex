% ═══════════════════════════════════════════════════════════════════════
% The Parity of Cubic Phases in Legendre Half-Intervals:
% A Negative Answer to the Cubic Zero-Skew Question
% Titan Project — Paper XIII — February 2026
% ═══════════════════════════════════════════════════════════════════════

\documentclass[11pt, a4paper]{article}

\usepackage[top=28mm, bottom=28mm, left=25mm, right=25mm]{geometry}
\usepackage[T1]{fontenc}
\usepackage{amsmath, amssymb, amsthm, mathtools}
\usepackage[dvipsnames]{xcolor}
\usepackage{enumitem}
\usepackage{booktabs}
\usepackage{hyperref}
\usepackage{float}

\newtheorem{theorem}{Theorem}[section]
\newtheorem{lemma}[theorem]{Lemma}
\newtheorem{corollary}[theorem]{Corollary}
\newtheorem{proposition}[theorem]{Proposition}
\theoremstyle{definition}
\newtheorem{definition}[theorem]{Definition}
\newtheorem{question}[theorem]{Open Question}
\theoremstyle{remark}
\newtheorem{remark}[theorem]{Remark}

\newcommand{\Leg}[2]{\left(\frac{#1}{#2}\right)}

\title{\textbf{The Parity of Cubic Phases in \\
Legendre Half-Intervals}}
\author{Ruqing Chen\\[4pt]
\small GUT Geoservice Inc., Montr\'eal, Canada\\[2pt]
\small \texttt{ruqing@hotmail.com}\\[2pt]
\small Repository:
\url{https://github.com/Ruqing1963/cubic-parity-half-intervals}}
\date{February 2026}

\begin{document}

\maketitle

\begin{abstract}
In \cite{ChenHigher}, we posed the question of whether a
``cubic Oppermann analogue'' exists: does the right half of
a Legendre interval exhibit zero cubic skewness under some
congruence condition?
We answer this question in the negative.
For $p = 2n - 1$ prime with $p \equiv 1 \pmod{3}$
(equivalently $p \equiv 1 \pmod{6}$),
the negation involution $x \mapsto p - x$
\emph{preserves} the cubic residue character
(since $\chi_3(-1) = 1$),
which forces every cubic phase count in the right half
to be even but does \emph{not} force equality among the
three phases.
In contrast, the left half contains a single unpaired
element---the midpoint $m = n(n-1) \equiv p - r \pmod{p}$,
whose involution partner is the puncture $r = 4^{-1}$---so
exactly one left-half phase count is odd and two are even.
This contrasts sharply with the quadratic case
\cite{ChenOpp}, where $\chi_2(-1) = -1$ for
$p \equiv 3 \pmod{4}$ causes the involution to
\emph{swap} QR and NR, producing exact zero skewness.
The symmetry breaking between $k = 2$ and $k = 3$
is thus a direct consequence of the parity of
$(p-1)/k$.
Verified for all 146 non-degenerate qualifying primes
up to $n = 1000$.
\end{abstract}

\bigskip

% ═══════════════════════════════════════════════════════════════════════
\section{Introduction}\label{sec:intro}
% ═══════════════════════════════════════════════════════════════════════

In \cite{ChenOpp}, we proved the Oppermann Parity Law:
for even $n$ (where $p = 2n - 1 \equiv 3 \pmod{4}$),
the right half of the Legendre interval has zero
quadratic residue skewness.
The proof rested on the negation involution
$x \mapsto p - x$, which maps QR to NR when
$\chi_2(-1) = -1$, forcing perfect pairing.

In \cite{ChenHigher}, we generalized the Spin Asymmetry
Theorem to $k$-th power residues for arbitrary $k$,
and posed Open Question~1: does the right half exhibit
zero \emph{cubic} skewness under some congruence
condition?

This note resolves the question.
The answer is \textbf{no}: the mechanism that produces
zero quadratic skewness fails for cubic residues,
for a precise algebraic reason.

% ═══════════════════════════════════════════════════════════════════════
\section{Setup and Notation}\label{sec:setup}
% ═══════════════════════════════════════════════════════════════════════

Let $n \ge 2$ with $p = 2n - 1$ prime and
$p \equiv 1 \pmod{3}$ (so $p \equiv 1 \pmod{6}$,
since $p$ is odd).
The Legendre interval interior is
$\mathcal{I}_n^\circ
= \{(n-1)^2 + 1, \ldots, n^2 - 1\}$,
which has $p - 1$ elements and constitutes a punctured
complete residue system modulo $p$ \cite{ChenSpin}.

The midpoint of the interval is
$m = n(n-1) = (n-1)^2 + (n-1)$.
We define:
\begin{align*}
    \mathcal{I}_L &= \{(n-1)^2 + 1, \ldots, m\}
    &&\text{(left half, $n - 1$ elements)}, \\
    \mathcal{I}_R &= \{m + 1, \ldots, n^2 - 1\}
    &&\text{(right half, $n - 1$ elements)}.
\end{align*}

The anchor (the unique multiple of $p$ in
$\mathcal{I}_n^\circ$) lies in exactly one half.
Let $\chi_3$ denote a cubic character modulo $p$.
For each phase
$\omega \in \{1, \zeta, \zeta^2\}$
(where $\zeta$ is a primitive cube root of unity
modulo $p$), define
$N_\omega^R = |\{x \in \mathcal{I}_R : \chi_3(x) = \omega,
\; p \nmid x\}|$.

% ═══════════════════════════════════════════════════════════════════════
\section{The Cubic Parity Theorem}\label{sec:theorem}
% ═══════════════════════════════════════════════════════════════════════

\begin{theorem}[Cubic parity: right half]
\label{thm:parity_right}
Let $p = 2n - 1$ be prime with $p \equiv 1 \pmod{6}$.
Then for each cubic phase $\omega$,
the right-half count $N_\omega^R$ is even.
\end{theorem}

\begin{proof}
The proof has three steps.

\textbf{Step 1: Phase preservation.}
Since $p \equiv 1 \pmod{6}$, the exponent
$(p-1)/3$ is even.
Therefore
\[
    \chi_3(-1) = (-1)^{(p-1)/3} = 1.
\]
For any $a \not\equiv 0 \pmod{p}$:
\begin{equation}\label{eq:preserve}
    \chi_3(p - a)
    = \chi_3(-a)
    = \chi_3(-1) \cdot \chi_3(a)
    = \chi_3(a).
\end{equation}

\textbf{Step 2: The involution on the right half.}
The right half consists of elements
$x \in \{m+1, \ldots, n^2 - 1\}$.
Their residues modulo $p$ form a set $S_R$.
We claim $S_R$ is closed under $a \mapsto p - a$.

To see this, observe that the full interior
$\mathcal{I}_n^\circ$ covers residues
$\{1, 2, \ldots, p-1\} \setminus \{r\}$
(where $r = n^2 \bmod p = 4^{-1} \bmod p$).
The negation involution partitions
$\{1, \ldots, p-1\}$ into $(p-1)/2$ orbits
$\{a, p-a\}$.
By \cite[Theorem~3.1]{ChenOpp}, the right half
$\mathcal{I}_R$ is closed under this involution:
if $x \in \mathcal{I}_R$ has residue $a$, then the
unique element of $\mathcal{I}_n^\circ$ with
residue $p - a$ also lies in $\mathcal{I}_R$.
When the anchor ($0 \bmod p$) lies in $\mathcal{I}_R$,
it is self-paired and excluded from the phase counts.

\textbf{Step 3: Orbit pairing.}
Since $p$ is odd, the equation $a \equiv p - a \pmod{p}$
has the unique solution $a = p/2$, which is not an
integer.
Hence every orbit $\{a, p-a\}$ in $S_R$ has size~$2$.
By~\eqref{eq:preserve}, both elements of each orbit
contribute to the \emph{same} cubic phase.
Therefore each $N_\omega^R$ is a sum of contributions
of~$2$, hence even.
\end{proof}

\begin{theorem}[Cubic parity: left half]
\label{thm:parity_left}
Under the same hypotheses, the left-half counts
$N_\omega^L$ satisfy: exactly two are even and one is
odd.
The unique odd count belongs to the cubic phase of the
midpoint $m = n(n-1)$.
\end{theorem}

\begin{proof}
The midpoint satisfies
$m = n^2 - n \equiv 4^{-1} - 2^{-1}
= -4^{-1} \equiv p - r \pmod{p}$.
The element with residue $p - r$ has lost its involution
partner $r$ (the puncture), so it is \emph{unpaired}.
Since $m$ is the last element of $\mathcal{I}_L$ and
$p - r$ always falls in $\mathcal{I}_L$, the left half
contains one unpaired element.
All other nonzero, non-anchor elements in $\mathcal{I}_L$
pair under $a \mapsto p - a$ within $\mathcal{I}_L$,
each pair contributing $2$ to one phase.
The single unpaired element $m$ contributes $1$ to its
phase $\chi_3(m) = \chi_3(p - r) = \chi_3(r)$
(by~\eqref{eq:preserve}).
Thus one phase has an odd count and the other two are
even.
\end{proof}

% ═══════════════════════════════════════════════════════════════════════
\section{The Negative Answer to the Zero-Skew Question}
\label{sec:negative}
% ═══════════════════════════════════════════════════════════════════════

\begin{theorem}[No cubic zero-skew]
\label{thm:no_zeroskew}
There is no congruence condition on $n$ that forces
$N_1^R = N_\zeta^R = N_{\zeta^2}^R$
for all qualifying primes.
\end{theorem}

\begin{proof}
The mechanism of \cite{ChenOpp} requires the involution
to \emph{exchange} phases, not preserve them.
In the quadratic case with $p \equiv 3 \pmod{4}$,
$\chi_2(-1) = -1$ forces each orbit to contain one QR
and one NR, yielding $N_{\mathrm{QR}}^R = N_{\mathrm{NR}}^R$.

In the cubic case, $\chi_3(-1) = 1$ forces each orbit
to lie \emph{within} a single phase.
This guarantees parity (Theorem~\ref{thm:parity_right})
but provides no constraint on the \emph{relative}
sizes of $N_1^R$, $N_\zeta^R$, $N_{\zeta^2}^R$.
Computationally, zero skewness never occurs:
among all 146 non-degenerate qualifying primes up to
$n = 1000$, the three phase counts are unequal in
every case (Table~\ref{tab:data}).
\end{proof}

\begin{remark}[The symmetry-breaking dichotomy]
The qualitative difference between the quadratic and
cubic half-interval behaviors is controlled by a single
bit: the parity of $(p-1)/k$.
\begin{center}
\renewcommand{\arraystretch}{1.2}
\begin{tabular}{@{}c c c c@{}}
\toprule
$k$ & Condition on $p$ & $(p{-}1)/k$ & Right-half behavior \\
\midrule
$2$ & $p \equiv 3 \pmod{4}$ & odd &
Involution \emph{swaps} $\Phi_0 \leftrightarrow \Phi_1$
$\Rightarrow$ zero skew \\
$2$ & $p \equiv 1 \pmod{4}$ & even &
Involution \emph{preserves}
$\Rightarrow$ even counts, nonzero skew \\
$3$ & $p \equiv 1 \pmod{3}$ & always even &
Involution \emph{preserves}
$\Rightarrow$ even counts, nonzero skew \\
\bottomrule
\end{tabular}
\end{center}
For $k = 3$, the ``swapping'' regime never occurs
because $p \equiv 1 \pmod{6}$ forces $(p-1)/3$ to
always be even.
\end{remark}

% ═══════════════════════════════════════════════════════════════════════
\section{General $k$ and the Quartic Case}
\label{sec:general}
% ═══════════════════════════════════════════════════════════════════════

\begin{proposition}[General parity/pairing criterion]
\label{prop:general}
Let $k \mid (p-1)$.
If $(p-1)/k$ is even, the negation involution preserves
all $k$-th power residue phases, and each right-half
phase count is even.
If $(p-1)/k$ is odd and $k$ is even, the involution maps
$\Phi_j \to \Phi_{j + k/2}$, pairing phases in
complementary orbits.
\end{proposition}

\begin{proof}
$\chi_k(-1) = (-1)^{(p-1)/k}$.
If this equals $1$ (even exponent), phase preservation
follows as in Theorem~\ref{thm:parity_right}.
If it equals $-1$ (odd exponent, requiring $k$ even),
then $\chi_k(p-a) = -\chi_k(a) = \zeta^{k/2} \chi_k(a)$,
shifting the phase index by $k/2$.
\end{proof}

For $k = 4$ with $p \equiv 5 \pmod{8}$ (so $(p-1)/4$ is
odd), the involution maps $\Phi_0 \leftrightarrow \Phi_2$
and $\Phi_1 \leftrightarrow \Phi_3$, forcing
$N_{\Phi_0}^R = N_{\Phi_2}^R$ and
$N_{\Phi_1}^R = N_{\Phi_3}^R$---a partial zero-skew
result for quartic residues.

% ═══════════════════════════════════════════════════════════════════════
\section{Computational Verification}\label{sec:data}
% ═══════════════════════════════════════════════════════════════════════

\begin{table}[H]
\centering
\caption{Right-half cubic phase counts for selected primes.
$\Phi_0, \Phi_1, \Phi_2$ are ordered by phase value
modulo $p$.
All counts are even; none are equal.}
\label{tab:data}
\medskip
\begin{tabular}{@{}r r r r r c@{}}
\toprule
$n$ & $p$ & $N_{\Phi_0}^R$ & $N_{\Phi_1}^R$ & $N_{\Phi_2}^R$ &
All even? \\
\midrule
10  & 19  &  2 &  4 &  2  & $\checkmark$ \\
16  & 31  &  6 &  2 &  6  & $\checkmark$ \\
22  & 43  &  8 &  4 &  8  & $\checkmark$ \\
34  & 67  & 14 &  8 & 10  & $\checkmark$ \\
40  & 79  & 16 &  8 & 14  & $\checkmark$ \\
100 & 199 & 24 & 38 & 36  & $\checkmark$ \\
205 & 409 & 60 & 66 & 78  & $\checkmark$ \\
505 & 1009& 168& 160& 176 & $\checkmark$ \\
1000& 1999& 314& 350& 334 & $\checkmark$ \\
\bottomrule
\end{tabular}
\end{table}

\begin{table}[H]
\centering
\caption{Left-half cubic phase counts.
Exactly one count is odd ($*$); it always matches
the phase of the midpoint $m = n(n-1)$.
Columns $\Phi_0, \Phi_1, \Phi_2$ use the same phase
ordering as Table~\ref{tab:data}, so column-wise sums
$N^L + N^R$ recover the full-interval deficit of
Paper~XII.}
\label{tab:left}
\medskip
\begin{tabular}{@{}r r r r r@{}}
\toprule
$n$ & $p$ & $N_{\Phi_0}^L$ & $N_{\Phi_1}^L$ & $N_{\Phi_2}^L$ \\
\midrule
10  & 19  &  4 & $1^*$ &  4  \\
16  & 31  & $3^*$ &  8 &  4  \\
22  & 43  & $5^*$ & 10 &  6  \\
34  & 67  &  8 & 14 & $11^*$ \\
40  & 79  & 10 & $17^*$& 12  \\
100 & 199 & 42 & 28 & $29^*$ \\
\bottomrule
\end{tabular}
\end{table}

\begin{table}[H]
\centering
\caption{Summary statistics ($n \le 1000$, non-degenerate cases).}
\label{tab:summary}
\medskip
\begin{tabular}{@{}l r@{}}
\toprule
Qualifying primes ($p \equiv 1 \pmod{6}$, $(p-1)/3 \ge 4$) & 146 \\
Right half: all three phase counts even & 146 (100\%) \\
Left half: exactly one odd count & 146 (100\%) \\
Left half: odd count in phase of midpoint $m$ & 146 (100\%) \\
Zero skew (right half) & 0 (0\%) \\
\bottomrule
\end{tabular}
\end{table}

% ═══════════════════════════════════════════════════════════════════════
\section{Discussion}\label{sec:discussion}
% ═══════════════════════════════════════════════════════════════════════

\subsection{Resolution of Open Question 1}

Open Question~1 of \cite{ChenHigher} asked whether a
cubic Oppermann analogue exists.
Theorem~\ref{thm:no_zeroskew} provides a definitive
negative answer: the cubic involution preserves
(rather than swaps) phases, yielding only a parity
constraint.
The zero-skew phenomenon of \cite{ChenOpp} is specific
to the quadratic case with $p \equiv 3 \pmod{4}$.

\subsection{The variance of cubic skewness}

Although phase counts are always even, their differences
can be large.
The phase variance
$V = \frac{1}{3}\sum_\omega (N_\omega^R - \bar{N})^2$
(where $\bar{N}$ is the mean count)
grows as $O(n)$, consistent with the
P\'olya--Vinogradov expectation for character sums
over intervals of length $\sim n$.
This linear growth is the natural behavior for
\emph{incomplete} character sums and requires no
deeper explanation.

\subsection{Hierarchy of half-interval constraints}

The Titan Project has now established a complete
hierarchy for the negation involution acting on
Legendre half-intervals:

\begin{enumerate}[nosep]
\item \textbf{Full interior} ($p - 1$ elements):
exact character sum $\sum \chi(x) = -\chi(r)$
\cite{ChenHigher}.
\item \textbf{Right half, $k = 2$, $p \equiv 3 \pmod{4}$}:
zero skewness, $N^+ = N^-$ \cite{ChenOpp}.
\item \textbf{Right half, $k = 3$, $p \equiv 1 \pmod{6}$}:
all phase counts even, but never equal (this paper,
Theorem~\ref{thm:parity_right}).
\item \textbf{Left half, $k = 3$, $p \equiv 1 \pmod{6}$}:
two even counts, one odd---the odd count belongs to the
phase of the midpoint $m$ (Theorem~\ref{thm:parity_left}).
\item \textbf{Right half, $k = 4$, $p \equiv 5 \pmod{8}$}:
complementary phase pairing
$N_{\Phi_0} = N_{\Phi_2}$, $N_{\Phi_1} = N_{\Phi_3}$
(Proposition~\ref{prop:general}).
\end{enumerate}

\subsection{Open questions}

\begin{question}[Quartic pairing verification]
For $p \equiv 5 \pmod{8}$, does the involution-forced
pairing $N_{\Phi_0}^R = N_{\Phi_2}^R$ hold
computationally for all qualifying primes?
\end{question}

\begin{question}[Skewness asymptotics]
Can the linear growth rate of the cubic phase variance
be made explicit, perhaps via the second moment of
cubic Dirichlet $L$-functions?
\end{question}

\subsection*{Data and code availability}

Verification scripts are available at
\url{https://github.com/Ruqing1963/cubic-parity-half-intervals}.

% ═══════════════════════════════════════════════════════════════════════
\begin{thebibliography}{99}

\bibitem{ChenSpin}
  R.~Chen,
  ``Algebraic rigidity and quadratic residue asymmetry
  in Legendre intervals,''
  Zenodo, 2026.
  \url{https://zenodo.org/records/18706876}

\bibitem{ChenOpp}
  R.~Chen,
  ``Oppermann's parity law: quadratic residue symmetry
  breaking in half-intervals via the negation involution,''
  Zenodo, 2026.
  \url{https://zenodo.org/records/18707265}

\bibitem{ChenHigher}
  R.~Chen,
  ``Higher-order residue deficits in Legendre intervals,''
  Zenodo, 2026.
  \url{https://zenodo.org/records/18714643}

\end{thebibliography}

\end{document}
